%%
%% This is file `sample-xelatex.tex',
%% generated with the docstrip utility.
%%
%% The original source files were:
%%
%% samples.dtx  (with options: `sigconf')
%%
%% IMPORTANT NOTICE:
%%
%% For the copyright see the source file.
%%
%% Any modified versions of this file must be renamed
%% with new filenames distinct from sample-xelatex.tex.
%%
%% For distribution of the original source see the terms
%% for copying and modification in the file samples.dtx.
%%
%% This generated file may be distributed as long as the
%% original source files, as listed above, are part of the
%% same distribution. (The sources need not necessarily be
%% in the same archive or directory.)
%%
%% The first command in your LaTeX source must be the \documentclass command.
\documentclass[sigconf]{acmart}

\usepackage{natbib}

%%
%% \BibTeX command to typeset BibTeX logo in the docs
\AtBeginDocument{%
  \providecommand\BibTeX{{%
    \normalfont B\kern-0.5em{\scshape i\kern-0.25em b}\kern-0.8em\TeX}}}

%% Rights management information.  This information is sent to you
%% when you complete the rights form.  These commands have SAMPLE
%% values in them; it is your responsibility as an author to replace
%% the commands and values with those provided to you when you
%% complete the rights form.
%\setcopyright{acmcopyright}
%\copyrightyear{2018}
%\acmYear{2018}
%\acmDOI{10.1145/1122445.1122456}

%% These commands are for a PROCEEDINGS abstract or paper.
%\acmConference[Woodstock '18]{Woodstock '18: ACM Symposium on Neural
%  Gaze Detection}{June 03--05, 2018}{Woodstock, NY}
%\acmBooktitle{Woodstock '18: ACM Symposium on Neural Gaze Detection,
%  June 03--05, 2018, Woodstock, NY}
%\acmPrice{15.00}
%\acmISBN{978-1-4503-XXXX-X/18/06}


%%
%% Submission ID.
%% Use this when submitting an article to a sponsored event. You'll
%% receive a unique submission ID from the organizers
%% of the event, and this ID should be used as the parameter to this command.
%%\acmSubmissionID{123-A56-BU3}

%%
%% The majority of ACM publications use numbered citations and
%% references.  The command \citestyle{authoryear} switches to the
%% "author year" style.
%%
%% If you are preparing content for an event
%% sponsored by ACM SIGGRAPH, you must use the "author year" style of
%% citations and references.
%% Uncommenting
%% the next command will enable that style.
%%\citestyle{acmauthoryear}

%%
%% end of the preamble, start of the body of the document source.
\begin{document}

%%
%% The "title" command has an optional parameter,
%% allowing the author to define a "short title" to be used in page headers.
\title{A Functional Computer Algebra\\with Some Examples in Kotlin}

%%
%% The "author" command and its associated commands are used to define
%% the authors and their affiliations.
%% Of note is the shared affiliation of the first two authors, and the
%% "authornote" and "authornotemark" commands
%% used to denote shared contribution to the research.

\author{Breandan Considine}
\affiliation{%
  \institution{McGill University}
}
\email{bre@ndan.co}

\author{Iaroslav Postovalov}
\affiliation{%
  \institution{JetBrains Research}
}
\email{postovalovya@gmail.com}

\author{Alexander Nozik}
\affiliation{%
  \institution{MIPT, JetBrains Research}
}
\email{altavir@gmail.com}

%%
%% By default, the full list of authors will be used in the page
%% headers. Often, this list is too long, and will overlap
%% other information printed in the page headers. This command allows
%% the author to define a more concise list
%% of authors' names for this purpose.
%\renewcommand{\shortauthors}{Trovato and Tobin, et al.}

%%
%% The abstract is a short summary of the work to be presented in the
%% article.
\begin{abstract}
We present a type-safe numerical tower starting with a generically typed algebra of groups, rings and fields, and show how to extend it to various domains, with examples in the Kotlin programming language. This hierarchy allows us to perform generic transformations on mathematical symbol trees. Some applications include linear algebra, automatic differentation and probabilistic programming.
\end{abstract}

%%
%% The code below is generated by the tool at http://dl.acm.org/ccs.cfm.
%% Please copy and paste the code instead of the example below.

\begin{CCSXML}
  <ccs2012>
  <concept>
  <concept_id>10002950.10003705</concept_id>
  <concept_desc>Mathematics of computing~Mathematical software</concept_desc>
  <concept_significance>500</concept_significance>
  </concept>
  </ccs2012>
\end{CCSXML}

\ccsdesc[500]{Mathematics of computing~Mathematical software}

%%
%% Keywords. The author(s) should pick words that accurately describe
%% the work being presented. Separate the keywords with commas.
\keywords{computer algebra, symbolic mathematics}

%%
%% This command processes the author and affiliation and title
%% information and builds the first part of the formatted document.
\maketitle

\section{Introduction}

% https://wiki.c2.com/?ExpressionProblem
% https://homepages.inf.ed.ac.uk/wadler/papers/expression/expression.txt

The expression problem~\citep{wadler1998expression} occurs when we want to implement some generic operator on multiple objects. They share the same interface, but have different implementations. Can we avoid writing it twice? Various solutions have been proposed~\citep{oliveira2012extensibility, oliveira2013feature, wang2016expression}.

Consider the following simplified example: ...

Can we avoid reimplementing it in multiple contexts?

\section{Context-Oriented Programming}

TODO: Alex

\section{Functional Programming}

TODO: Iaroslav?

\section{Symbolic Differentiation}

TODO: Breandan

\section{Computational Geometry}

TODO: Alex

\section{Probabilistic Programming}

TODO: Breandan

\section{Simplification}

TODO: Breandan

%%
%% The next two lines define the bibliography style to be used, and
%% the bibliography file.
\bibliographystyle{ACM-Reference-Format}
\bibliography{issac}

%%
%% If your work has an appendix, this is the place to put it.

\appendix

\section{Appendix}

\subsection{Part One}

Lorem ipsum dolor sit amet, consectetur adipiscing elit.

\subsection{Part Two}

Etiam commodo feugiat nisl pulvinar pellentesque.

\section{Online Resources}

\url{https://github.com/mipt-npm/kmath}

\url{https://github.com/breandan/kotlingrad}

\end{document}
\endinput
%%
%% End of file `sample-xelatex.tex'.