%%%%%%%%%%%%%%%%%%%%%%%%%%%%%%%%%%%%%%%%%%%
%
% myfig
%
% \myfig - replacement for \figure
% necessary, since in multicol-environment
% \figure won't work
%
%%%%%%%%%%%%%%%%%%%%%%%%%%%%%%%%%%%%%%%%%%%

\newcommand{\myfig}[3][0]{
\begin{center}
    \vspace{1.5cm}
    \includegraphics[width=#3\hsize,angle=#1]{#2}
    \nobreak\medskip
\end{center}}

%%%%%%%%%%%%%%%%%%%%%%%%%%%%%%%%%%%%%%%%%%%
%
% mycaption
%
% \mycaption - replacement for \caption
% necessary, since in multicol-environment \figure and
% therefore \caption won't work
%
%%%%%%%%%%%%%%%%%%%%%%%%%%%%%%%%%%%%%%%%%%%

%\newcounter{figure}
\setcounter{figure}{1}
\newcommand{\mycaption}[1]{
\vspace{0.5cm}
\begin{quote}
{{\sc Figure} \arabic{figure}: #1}
\end{quote}
\vspace{1cm}
\stepcounter{figure}
}

%%%%%%%%%%%%%%%%%%%%%%%%%%%%%%%%%%%%%%%%%%%
%
% Some standard colours
%
%%%%%%%%%%%%%%%%%%%%%%%%%%%%%%%%%%%%%%%%%%%

\definecolor{camlightblue}{rgb}{0.601 , 0.8, 1}
\definecolor{camdarkblue}{rgb}{0, 0.203, 0.402}
\definecolor{camred}{rgb}{1, 0.203, 0}
\definecolor{camyellow}{rgb}{1, 0.8, 0}
\definecolor{lightblue}{rgb}{0, 0, 0.80}
\definecolor{white}{rgb}{1, 1, 1}
\definecolor{whiteblue}{rgb}{0.80, 0.80, 1}

%%%%%%%%%%%%%%%%%%%%%%%%%%%%%%%%%%%%%%%%%%%
%
% Some look and feel definitions
%
%%%%%%%%%%%%%%%%%%%%%%%%%%%%%%%%%%%%%%%%%%%

\setlength{\columnsep}{0.03\textwidth}
\setlength{\columnseprule}{0.0018\textwidth}
\setlength{\parindent}{0.0cm}

%%%%%%%%%%%%%%%%%%%%%%%%%%%%%%%%%%%%%%%%%%%
%
% \mysection - replacement for \section*
%
% Puts a pretty box around some text
% TODO - any other thoughts for what this box should look like
%
%%%%%%%%%%%%%%%%%%%%%%%%%%%%%%%%%%%%%%%%%%%

\tikzstyle{mysection} = [rectangle,
draw=none,
shade,
outer color=camlightblue!30,
inner color=camlightblue!30,
text width=0.965\columnwidth,
text centered,
rounded corners=20pt,
minimum height=0.09\columnwidth]

\newcommand{\mysection}[1]
{
\begin{center}
    \begin{tikzpicture}
        \node[mysection] {\sffamily\bfseries\LARGE#1};
    \end{tikzpicture}
\end{center}
}

%%%%%%%%%%%%%%%%%%%%%%%%%%%%%%%%%%%%%%%%%%%
%
% Set the font
%
% TODO - Not sure what a canonical choice is - feel free to modify
%
%%%%%%%%%%%%%%%%%%%%%%%%%%%%%%%%%%%%%%%%%%%

\renewcommand{\familydefault}{cmss}
\sffamily

%%%%%%%%%%%%%%%%%%%%%%%%%%%%%%%%%%%%%%%%%%%%%%%%%%%%
%%%               Background                     %%%
%%%%%%%%%%%%%%%%%%%%%%%%%%%%%%%%%%%%%%%%%%%%%%%%%%%%

\newcommand{\background}[3]{
%\definecolor{cgradbegin}{#1}
%\definecolor{cgradend}{#2}
% \psframe[fillstyle=gradient,gradend=cgradend,
% gradbegin=cgradbegin,gradmidpoint=#3](0.,0.)(1.\textwidth,-1.\textheight)
}




%%%%%%%%%%%%%%%%%%%%%%%%%%%%%%%%%%%%%%%%%%%%%%%%%%%%
%%%                pcolumn                       %%%
%%%%%%%%%%%%%%%%%%%%%%%%%%%%%%%%%%%%%%%%%%%%%%%%%%%%

\newenvironment{pcolumn}[1]{
\begin{minipage}{#1\textwidth}
\begin{center}
}{
\end{center}
\end{minipage}
}



%%%%%%%%%%%%%%%%%%%%%%%%%%%%%%%%%%%%%%%%%%%%%%%%%%%%
%%%                pbox                          %%%
%%%%%%%%%%%%%%%%%%%%%%%%%%%%%%%%%%%%%%%%%%%%%%%%%%%%

\definecolor{lcolor}{rgb}{0, 0, 0.80}
\definecolor{gcolor1}{rgb}{1, 1, 1}
\definecolor{gcolor2}{rgb}{.80, .80, 1}

% \def\fc{fillcolor}
% \def\getfc #1=#2\par{\def\ffc{#1} \ifx\ffc\fc #2\fi}
% \def\getfillcolor #1,#2\par{\getfc #1\par \getfc #2\par}

%  \newcommand{\psshadowbox}[2]{%[2][magenta]{
%      \fbox{Input arg: #1}
%      \fbox{#1}
%      \fbox {\getfillcolor #1\par}
%      \def\col{\getfillcolor #1\par}

%      \let\coll=\col
%       \coll
%     \colorbox{\col}{#2}
%       \mbox
%   \coloredshadowbox{black}{\coll}{#2}
%   }

\newcommand{\pbox}[4]{
%\psshadowbox[#3]{
%\fbox{
\mbox{
\begin{minipage}[t][#2][t]{#1}
#4
\end{minipage}
}%}
}

%%%%%%%%%%%%%%%%%%%%%%%%%%%%%%%%%%%%%%%%%%%
%
% Poster environment
%
% Centres everything and can be used to define the width of the content
%
%%%%%%%%%%%%%%%%%%%%%%%%%%%%%%%%%%%%%%%%%%%

\newenvironment{poster}{
\begin{center}
\begin{minipage}[c]{\textwidth}
}{
\end{minipage}
\end{center}
}

\def\newarrow{\mbox{\begin{tikzpicture}
\useasboundingbox{(-3pt,-4.5pt) rectangle (19pt,1pt)};
\draw[->] (0,-0.07)--(17pt,-0.07);\end{tikzpicture}}}

%%%%%%%%%%%%%%%%%%%%%%%%%%%%%%%%%%%%%%%%%%%
%
% Bottom box
%
%%%%%%%%%%%%%%%%%%%%%%%%%%%%%%%%%%%%%%%%%%%

\newlength{\bottomboxheight}
\setlength{\bottomboxheight}{0.1\paperheight}

\newcommand{\bottombox}[1]{\vfill
\noindent\colorbox{white}{
\begin{minipage}[c][\bottomboxheight][c]{\textwidth}
\centering
\begin{minipage}{0.9\textwidth}
\vfill{

\fontsizesection\color{black}
#1
}

\end{minipage}
\end{minipage}

}
}

%% Bottom box logo
\newcommand{\bottomboxlogo}[2][width=\textwidth]{
\begin{minipage}[c][\bottomboxheight][c]{0.3\textwidth}
\raggedleft\includegraphics[#1]{#2}
\end{minipage}
}

\newcommand{\bottomboxlogoleft}[2][width=\textwidth]{
\begin{minipage}[l][\bottomboxheight][c]{0.3\textwidth}
\raggedleft\includegraphics[#1]{#2}
\end{minipage}
}

%%%%%%%%%%%%%%%%%%%%%%%%%%%%%%%%%%%%%%%%%%%
%
% Highlighting
%
%%%%%%%%%%%%%%%%%%%%%%%%%%%%%%%%%%%%%%%%%%%

\definecolor{slightgray}{rgb}{0.90, 0.90, 0.90}

\usepackage{soul}
\makeatletter
\def\SOUL@hlpreamble{%
\setul{}{3.0ex}%
\let\SOUL@stcolor\SOUL@hlcolor%
\SOUL@stpreamble%
}
\makeatother

\newcommand{\inline}[1]{%
\begingroup%
\sethlcolor{slightgray}%
\hl{\ttfamily\small #1}%
\endgroup
}

\newcommand{\tinline}[1]{%
\begingroup%
\sethlcolor{slightgray}%
\hl{\ttfamily #1}%
\endgroup
}

%%%%%%%%%%%%%%%%%%%%%%%%%%%%%%%%%%%%%%%%%%%
%
% Kotlin syntax highlighting
%
%%%%%%%%%%%%%%%%%%%%%%%%%%%%%%%%%%%%%%%%%%%

\usepackage[skins,breakable,listings]{tcolorbox}

\usepackage[dvipsnames]{xcolor}
\usepackage[table]{xcolor}
\lstdefinelanguage{kotlin}{
comment=[l]{//},
commentstyle={\color{gray}\ttfamily},
emph={delegate, filter, firstOrNull, forEach, it, lazy, mapNotNull, println, @Repeat, return@},
emphstyle={\color{OrangeRed}},
identifierstyle=\color{black},
keywords={abstract, actual, as, as?, break, by, class, companion, continue, data, do, dynamic, else, enum, expect, false, final, for, fun, get, if, import, in, infix, interface, internal, is, null, object, open, operator, override, package, private, public, return, sealed, set, super, suspend, this, throw, true, try, typealias, val, var, vararg, when, where, while, tailrec, reified},
keywordstyle={\color{blue}\bfseries},
morecomment=[s]{/*}{*/},
morestring=[b]",
morestring=[s]{"""*}{*"""},
ndkeywords={@Deprecated, @JvmField, @JvmName, @JvmOverloads, @JvmStatic, @JvmSynthetic, Array, Byte, Double, Float, Int, Integer, Iterable, Long, Runnable, Short, String},
ndkeywordstyle={\color{BurntOrange}\bfseries},
sensitive=true,
stringstyle={\color{ForestGreen}\ttfamily},
literate={`}{{\char0}}1
}

%%%%%%%%%%%%%%%%%%%%%%%%%%%%%%%%%%%%%%%%%%%
%
% Color boxes
%
%%%%%%%%%%%%%%%%%%%%%%%%%%%%%%%%%%%%%%%%%%%

\tcbset{
enhanced jigsaw,
breakable,
listing only,
boxsep=-1pt,
top=-1pt,
bottom=-0.5pt,
right=-0.5pt,
overlay first={
\node[black!50] (S) at (frame.south) {\Large\ding{34}};
\draw[dashed,black!50] (frame.south west) -- (S) -- (frame.south east);
},
overlay middle={
\node[black!50] (S) at (frame.south) {\Large\ding{34}};
\draw[dashed,black!50] (frame.south west) -- (S) -- (frame.south east);
\node[black!50] (S) at (frame.north) {\Large\ding{34}};
\draw[dashed,black!50] (frame.north west) -- (S) -- (frame.north east);
},
overlay last={
\node[black!50] (S) at (frame.north) {\Large\ding{34}};
\draw[dashed,black!50] (frame.north west) -- (S) -- (frame.north east);
},
before={\par\vspace{10pt}},
after={\par\vspace{\parskip}\noindent}
}

\newtcblisting{kotlinlisting}[1][]{%
width=20.5cm,
left=20pt,
top=5pt,
listing options={
language=kotlin,
basicstyle=\ttfamily\normalsize,
%numberstyle=\footnotesize,
showstringspaces=false,
tabsize=2,
breaklines=true,
numbers=none,
inputencoding=utf8,
escapeinside={(*}{*)},
#1
},
underlay unbroken and first={%
\path[draw=none] (interior.north west) rectangle node[white]{\includegraphics[width=10mm]{../figures/kotlin_file.png}} ([xshift=-18mm,yshift=-20mm]interior.north west);
}
}

\newtcblisting{pythonlisting}[1][]{
width=17cm,
left=20pt,
top=5pt,
listing options={
language=Python,
basicstyle=\ttfamily\normalsize,
upquote=true,
breaklines=true,
showstringspaces=false,
keywordstyle=\color{blue}\bfseries,
escapeinside={(*}{*)},
#1
},
fonttitle=\ttfamily\small,
underlay unbroken and first={
\path[draw=none] (interior.north west) rectangle node[white]{\includegraphics[width=10mm]{../figures/python_icon.png}} ([xshift=-18mm,yshift=-20mm]interior.north west);
}
}

% Imitate syntax error
\usepackage{ulem}
\makeatletter
\def\uwave{\bgroup \markoverwith{\lower7.5\p@\hbox{\sixly \textcolor{red}{\char58}}}\ULon}
\font\sixly=lasy6 % does not re-load if already loaded, so no memory problem.
\makeatother

\usepackage{tikz}
\usepackage[skins,breakable,listings]{tcolorbox}
\usepackage{pgfplots}
\usepackage{tikz-qtree}
\usepackage{graphicx}