\documentclass{beamer}
\mode<presentation> { \usetheme{Madrid} }

\title{Kotlin\texorpdfstring{$\nabla$}{}}
\subtitle{Differentiable functional programming with algebraic data types}
\author{Breandan Considine}
\institute[UdeM]{
Universit\'e de Montr\'eal \\
\medskip
\textit{breandan.considine@umontreal.ca}
}
\date{\today}

\begin{document}
    \begin{frame}
        \titlepage
    \end{frame}

    \begin{frame}
        \frametitle{Overview}
        \tableofcontents
    \end{frame}

    \section{A Short History of Computing Derivatives}\label{sec:first-section}

    %------------------------------------------------------------------------------------------------

    \begin{frame}
        \frametitle{Numerical differentiation}
        \begin{itemize}
            \item Many mathematical formula have discrete, numerical representations
            \item But numerical values can be an \textit{inexact} representation of math
            \item Long calculations on primitives are susceptible to rounding errors
            \item Bag of tricks for discrete approximation and numerical stability:
            \begin{itemize}
                \item Fourier, Chebyshev, Lagrange
                \item Arbitrary precision arithmetic
                \item Kahan summation algorithm
                \item log-sum-exp trick
            \end{itemize}
        \end{itemize}
    \end{frame}

    %------------------------------------------------------------------------------------------------

    \begin{frame}
        \frametitle{Symbolic differentiation}
        \begin{itemize}
            \item What about evaluating functions symbolically?
            \item Computer algebra systems for manipulate symbolic formulas
        \end{itemize}
    \end{frame}

    %------------------------------------------------------------------------------------------------

    \begin{frame}
        \frametitle{Automatic differentiation}
        \begin{itemize}
            \item Derivatives can be calculated automatically? (Wengert, 1964)
            \item Code as an \textit{exact} symbolic representation of functions
            \item To reason about code we need the ability to treat \textit{code as data}:
            \begin{itemize}
                \item Reflection and metaprogramming
                \item Domain specific languages
                \item First-class functions
            \end{itemize}
        \end{itemize}
    \end{frame}

    %------------------------------------------------------------------------------------------------

    \begin{frame}
        \frametitle{Differentiable [functional] programming}
        \begin{itemize}
            \item What is a program, but a series of arithmetic operations?
            \item What are arithmetic operations but syntactic sugar for functions?
            \item Functions can be composed of other functions or chained in sequence
            \item High school calculus gives us rules for differentiating function chains
            \item Pearlmutter \& Siskind teach us AD is possible just using FP (2016)
            \item Wang, Rompf, et al. show us this is possible \textit{without a tape}! (2018)
        \end{itemize}
    \end{frame}

    %------------------------------------------------------------------------------------------------

    \begin{frame}
        \frametitle{Differentiable programming with algebraic types}
        \begin{itemize}
            \item How do we get autocompletion and static analysis?
            \item There is an abstract algebra for tensor manipulations
            \item Allows us to preserve symmetries that are not obvious
            \item Can be encoded using OOP and parametric polymorphism
            \item \href{https://arxiv.org/pdf/1610.07690.pdf}{Operational Calculus for Differentiable Programming}
        \end{itemize}
    \end{frame}

    %------------------------------------------------------------------------------------------------

    \section{Architectural Overview}\label{sec:second-section}

    \begin{frame}
        \frametitle{Architecture}
    \end{frame}

    \section{Usage}\label{sec:third-section}

    \begin{frame}
        \frametitle{Usage}
    \end{frame}

    \section{Future Plans}\label{sec:fourth-section}

    \begin{frame}
        \frametitle{Roadmap}
    \end{frame}

\end{document}

% %----------------------------------------------------------------------------------------
% %	PRESENTATION SLIDES
% %----------------------------------------------------------------------------------------

% %------------------------------------------------
% \section{First Section}

% %------------------------------------------------

% \subsection{Subsection Example} % A subsection can be created just before a set of slides with a common theme to further break down your presentation into chunks

% \begin{frame}
% \frametitle{Paragraphs of Text}
% Sed iaculis dapibus gravida. Morbi sed tortor erat, nec interdum arcu. Sed id lorem lectus. Quisque viverra augue id sem ornare non aliquam nibh tristique. Aenean in ligula nisl. Nulla sed tellus ipsum. Donec vestibulum ligula non lorem vulputate fermentum accumsan neque mollis.\\~\\

% Sed diam enim, sagittis nec condimentum sit amet, ullamcorper sit amet libero. Aliquam vel dui orci, a porta odio. Nullam id suscipit ipsum. Aenean lobortis commodo sem, ut commodo leo gravida vitae. Pellentesque vehicula ante iaculis arcu pretium rutrum eget sit amet purus. Integer ornare nulla quis neque ultrices lobortis. Vestibulum ultrices tincidunt libero, quis commodo erat ullamcorper id.
% \end{frame}

% %------------------------------------------------

% \begin{frame}
% \frametitle{Bullet Points}
% \begin{itemize}
% \item Lorem ipsum dolor sit amet, consectetur adipiscing elit
% \item Aliquam blandit faucibus nisi, sit amet dapibus enim tempus eu
% \item Nulla commodo, erat quis gravida posuere, elit lacus lobortis est, quis porttitor odio mauris at libero
% \item Nam cursus est eget velit posuere pellentesque
% \item Vestibulum faucibus velit a augue condimentum quis convallis nulla gravida
% \end{itemize}
% \end{frame}

% %------------------------------------------------

% \begin{frame}
% \frametitle{Blocks of Highlighted Text}
% \begin{block}{Block 1}
% Lorem ipsum dolor sit amet, consectetur adipiscing elit. Integer lectus nisl, ultricies in feugiat rutrum, porttitor sit amet augue. Aliquam ut tortor mauris. Sed volutpat ante purus, quis accumsan dolor.
% \end{block}

% \begin{block}{Block 2}
% Pellentesque sed tellus purus. Class aptent taciti sociosqu ad litora torquent per conubia nostra, per inceptos himenaeos. Vestibulum quis magna at risus dictum tempor eu vitae velit.
% \end{block}

% \begin{block}{Block 3}
% Suspendisse tincidunt sagittis gravida. Curabitur condimentum, enim sed venenatis rutrum, ipsum neque consectetur orci, sed blandit justo nisi ac lacus.
% \end{block}
% \end{frame}

% %------------------------------------------------

% \begin{frame}
% \frametitle{Multiple Columns}
% \begin{columns}[c] % The "c" option specifies centered vertical alignment while the "t" option is used for top vertical alignment

% \column{.45\textwidth} % Left column and width
% \textbf{Heading}
% \begin{enumerate}
% \item Statement
% \item Explanation
% \item Example
% \end{enumerate}

% \column{.5\textwidth} % Right column and width
% Lorem ipsum dolor sit amet, consectetur adipiscing elit. Integer lectus nisl, ultricies in feugiat rutrum, porttitor sit amet augue. Aliquam ut tortor mauris. Sed volutpat ante purus, quis accumsan dolor.

% \end{columns}
% \end{frame}

% %------------------------------------------------
% \section{Second Section}
% %------------------------------------------------

% \begin{frame}
% \frametitle{Table}
% \begin{table}
% \begin{tabular}{l l l}
% \toprule
% \textbf{Treatments} & \textbf{Response 1} & \textbf{Response 2}\\
% \midrule
% Treatment 1 & 0.0003262 & 0.562 \\
% Treatment 2 & 0.0015681 & 0.910 \\
% Treatment 3 & 0.0009271 & 0.296 \\
% \bottomrule
% \end{tabular}
% \caption{Table caption}
% \end{table}
% \end{frame}

% %------------------------------------------------

% \begin{frame}
% \frametitle{Theorem}
% \begin{theorem}[Mass--energy equivalence]
% $E = mc^2$
% \end{theorem}
% \end{frame}

% %------------------------------------------------

% \begin{frame}[fragile] % Need to use the fragile option when verbatim is used in the slide
% \frametitle{Verbatim}
% \begin{example}[Theorem Slide Code]
% \begin{verbatim}
% \begin{frame}
% \frametitle{Theorem}
% \begin{theorem}[Mass--energy equivalence]
% $E = mc^2$
% \end{theorem}
% \end{frame}\end{verbatim}
% \end{example}
% \end{frame}

% %------------------------------------------------

% \begin{frame}
% \frametitle{Figure}
% Uncomment the code on this slide to include your own image from the same directory as the template .TeX file.
% %\begin{figure}
% %\includegraphics[width=0.8\linewidth]{test}
% %\end{figure}
% \end{frame}

% %------------------------------------------------

% \begin{frame}[fragile] % Need to use the fragile option when verbatim is used in the slide
% \frametitle{Citation}
% An example of the \verb|\cite| command to cite within the presentation:\\~

% This statement requires citation \cite{p1}.
% \end{frame}

% %------------------------------------------------

% \begin{frame}
% \frametitle{References}
% \footnotesize{
% \begin{thebibliography}{99} % Beamer does not support BibTeX so references must be inserted manually as below
% \bibitem[Smith, 2012]{p1} John Smith (2012)
% \newblock Title of the publication
% \newblock \emph{Journal Name} 12(3), 45 -- 678.
% \end{thebibliography}
% }
% \end{frame}

% %------------------------------------------------

% \begin{frame}
% \Huge{\centerline{The End}}
% \end{frame}

% %----------------------------------------------------------------------------------------